\documentclass{article}
\usepackage[ae,hyper]{Rd}
\begin{document}
\Header{edist}{E-distance}
\keyword{multivariate}{edist}
\keyword{cluster}{edist}
\keyword{nonparametric}{edist}
\begin{Description}\relax
Returns the E-distances (energy statistics) between clusters.
\end{Description}
\begin{Usage}
\begin{verbatim}
 edist(x, sizes, distance=FALSE, ix = 1:sum(sizes))
\end{verbatim}
\end{Usage}
\begin{Arguments}
\begin{ldescription}
\item[\code{x}] data matrix of pooled sample or Euclidean distances
\item[\code{sizes}] vector of sample sizes
\item[\code{distance}] logical: if TRUE, x is a distance matrix
\item[\code{ix}] a permutation of the row indices of x 
\end{ldescription}
\end{Arguments}
\begin{Details}\relax
A vector containing the pairwise two-sample multivariate 
\eqn{\mathcal{E}}{E}-statistics for comparing clusters or samples is returned. 
The e-distance between clusters is computed from the original pooled data, 
stacked in matrix \code{x} where each row is a multivariate observation, or 
from the distance matrix \code{x} of the original data, or distance object 
returned by \code{dist}. The first \code{sizes[1]} rows of the original data 
matrix are the first sample, the next \code{sizes[2]} rows are the second 
sample, etc. The permutation vector \code{ix} may be used to obtain
e-distances corresponding to a clustering solution at a given level in
the hierarchy.

The e-distance between two clusters \eqn{C_i, C_j}{}
of size \eqn{n_i, n_j}{} 
proposed by Szekely and Rizzo (2003ab)
is the e-distance \eqn{e(C_i,C_j)}{}, defined by
\deqn{e(C_i,C_j)=\frac{n_i n_j}{n_i+n_j}[2M_{ij}-M_{ii}-M_{jj}],
}{e(S_i, S_j) = (n_i n_j)(n_i+n_j)[2M_(ij)-M_(ii)-M_(jj)],}
where
\deqn{M_{ij}=\frac{1}{n_i n_j}\sum_{p=1}^{n_i} \sum_{q=1}^{n_j}
\|X_{ip}-X_{jq}\|,}{M_{ij} = 1/(n_i n_j) sum[1:n_i, 1:n_j] ||X_(ip) - X_(jq)||,}
\eqn{\|\cdot\|}{|| ||} denotes Euclidean norm, and \eqn{X_{ip}}{X_(ip)} denotes the p-th observation in the i-th cluster.\end{Details}
\begin{Value}
A object of class \code{dist} containing the lower triangle of the
e-distance matrix of cluster distances corresponding to the permutation 
of indices \code{ix} is returned.\end{Value}
\begin{Author}\relax
Maria Rizzo \email{rizzo@math.ohiou.edu}
\end{Author}
\begin{References}\relax
Szekely, G. J. and Rizzo, M. L. (2003a) Hierarchical Clustering
via Joint Between-Within Distances, submitted.

Szekely, G. J. and Rizzo, M. L. (2003b) Testing for Equal
Distributions in High Dimension, submitted.

Szekely, G. J. (2000) \eqn{\mathcal{E}}{E}-statistics: Energy of 
Statistical Samples, preprint.\end{References}
\begin{SeeAlso}\relax
\code{\Link{energy.hclust}}
\code{\Link{eqdist.etest}} \code{\Link{ksample.e}}
\end{SeeAlso}
\begin{Examples}
\begin{ExampleCode}
 ## compute e-distances for 3 samples of iris data
 data(iris)
 edist(iris[,1:4], c(50,50,50))

 ## compute e-distances from a distance object
 data(iris)
 edist(dist(iris[,1:4]), c(50, 50, 50), distance=TRUE)

 ## compute e-distances from a distance matrix
 data(iris)
 d <- as.matrix(dist(iris[,1:4]))
 edist(d, c(50, 50, 50), distance=TRUE) 
\end{ExampleCode}
\end{Examples}

\Header{energy.hclust}{Hierarchical Clustering by Minimum (Energy) E-distance}
\keyword{multivariate}{energy.hclust}
\keyword{cluster}{energy.hclust}
\begin{Description}\relax
Performs hierarchical clustering on a set of Euclidean distance 
dissimilarities by minimum (energy) E-distance method.
\end{Description}
\begin{Usage}
\begin{verbatim}
    energy.hclust(dst)
\end{verbatim}
\end{Usage}
\begin{Arguments}
\begin{ldescription}
\item[\code{dst}] dissimilarity object produced by \code{dist} with
\code{method=euclidean}, or lower triangle of distance
matrix as vector in column order. If \code{dst} is a square
matrix, the lower triangle is interpreted as a vector of
distances.
\end{ldescription}
\end{Arguments}
\begin{Details}\relax
This function performs agglomerative hierarchical cluster analysis
based on the pairwise distances between sample elements in \code{dst}.
Initially, each of the n singletons is a cluster. At each of n-1 steps, the 
procedure merges the pair of clusters with minimum e-distance. 
The e-distance
between two clusters \eqn{C_i, C_j}{} of sizes \eqn{n_i, n_j}{} is given by
\deqn{e(C_i, C_j)=\frac{n_i n_j}{n_i+n_j}[2M_{ij}-M_{ii}-M_{jj}],
}{}
where
\deqn{M_{ij}=\frac{1}{n_i n_j}\sum_{p=1}^{n_i} \sum_{q=1}^{n_j}
\|X_{ip}-X_{jq}\|,}{M_{ij} = 1/(n_i n_j) sum[1:n_i, 1:n_j] ||X_(ip) - X_(jq)||,}
\eqn{\|\cdot\|}{|| ||} denotes Euclidean norm, and \eqn{X_{ip}}{X_(ip)} denotes the p-th observation in the i-th cluster.  

The return value is an object of class \code{hclust}, so \code{hclust}
methods such as print or plot methods, \code{plclust}, and \code{cutree}
are available. See the documentation for \code{hclust}.

The e-distance measures both the heterogeneity between clusters and the
homogeneity within clusters. E-clustering is particularly effective in
high dimension, and is more effective than some standard hierarchical
methods when clusters have equal means (see example below).
For other advantages see the references.\end{Details}
\begin{Value}
An object of class \code{hclust} which describes the tree produced by
the clustering process. The object is a list with components: 
\begin{ldescription}
\item[\code{merge:}] an n-1 by 2 matrix, where row i of \code{merge} describes the
merging of clusters at step i of the clustering. If an element j in the
row is negative, then observation -j was merged at this
stage. If j is positive then the merge was with the cluster
formed at the (earlier) stage j of the algorithm.
\item[\code{height:}] the clustering height: a vector of n-1 non-decreasing
real numbers (the e-distance between merging clusters)
\item[\code{order:}] a vector giving a permutation of the indices of 
original observations suitable for plotting, in the sense that a 
cluster plot using this ordering and matrix \code{merge} will not have 
crossings of the branches.
\item[\code{labels:}] labels for each of the objects being clustered.
\item[\code{call:}] the call which produced the result.
\item[\code{method:}] the cluster method that has been used (e-distance).
\item[\code{dist.method:}] the distance that has been used to create \code{dst}.
\end{ldescription}
\end{Value}
\begin{Author}\relax
Maria Rizzo
\end{Author}
\begin{References}\relax
Szekely, G. J. and Rizzo, M. L. (2003) Hierarchical Clustering
via Joint Between-Within Distances, submitted.

Szekely, G. J. and Rizzo, M. L. (2003) Testing for Equal
Distributions in High Dimension, submitted.

Szekely, G. J. (2000) \eqn{\mathcal E}{E}-statistics: Energy of
Statistical Samples, preprint.\end{References}
\begin{SeeAlso}\relax
\code{\Link{edist}} \code{\Link{ksample.e}} \code{hclust}
\end{SeeAlso}
\begin{Examples}
\begin{ExampleCode}
   ## Not run: 
   
   library(cluster)
   data(animals)
   plot(energy.hclust(dist(animals)))
   
## End(Not run)
   
   library(mva)
   data(USArrests)
   ecl <- energy.hclust(dist(USArrests))
   print(ecl)    
   plot(ecl)
   cutree(ecl, k=3)
   cutree(ecl, h=150)
   
   ## compare performance of e-clustering, Ward's method, group average method
   ## when sampled populations have equal means: n=200, d=5, two groups
   z <- rbind(matrix(rnorm(1000), nrow=200), matrix(rnorm(1000, 0, 5), nrow=200))
   g <- c(rep(1, 200), rep(2, 200))
   d <- dist(z)
   e <- energy.hclust(d)
   a <- hclust(d, method="average")
   w <- hclust(d^2, method="ward")
   list("E" = table(cutree(e, k=2) == g), "Ward" = table(cutree(w, k=2) == g),
        "Avg" = table(cutree(a, k=2) == g))
 \end{ExampleCode}
\end{Examples}

\Header{eqdist.etest}{Multisample E-statistic (Energy) Test of Equal Distributions}
\keyword{multivariate}{eqdist.etest}
\keyword{htest}{eqdist.etest}
\keyword{nonparametric}{eqdist.etest}
\begin{Description}\relax
Performs the nonparametric multisample E-statistic (energy) test
for equality of multivariate distributions.
\end{Description}
\begin{Usage}
\begin{verbatim}
 eqdist.etest(x, sizes, distance = FALSE, 
              incomplete = FALSE, N = 100, R = 999)
\end{verbatim}
\end{Usage}
\begin{Arguments}
\begin{ldescription}
\item[\code{x}] data matrix of pooled sample
\item[\code{sizes}] vector of sample sizes
\item[\code{distance}] logical: if TRUE, first argument is a distance matrix
\item[\code{incomplete}] logical: if TRUE, compute incomplete \eqn{\mathcal{E}}{E}-statistics
\item[\code{N}] incomplete sample size
\item[\code{R}] number of bootstrap replicates 
\end{ldescription}
\end{Arguments}
\begin{Details}\relax
The k-sample multivariate \eqn{\mathcal{E}}{E}-test of equal distributions
is performed. The statistic is computed from the original
pooled samples, stacked in matrix \code{x} where each row
is a multivariate observation, or the corresponding distance matrix. The
first \code{sizes[1]} rows of \code{x} are the first sample, the next
\code{sizes[2]} rows of \code{x} are the second sample, etc.

The test is implemented by nonparametric bootstrap, an approximate 
permutation test with \code{R} replicates.

The definition of the multisample \eqn{\mathcal{E}}{E}-statistic is given in the 
\code{\Link{ksample.e}} documentation.

If \code{incomplete==TRUE}, incomplete \eqn{\mathcal{E}}{E}-statistics (which are
incomplete V-statistics) are computed. That is, at most
\code{N} observations from each sample are used, by sampling without replacement 
as needed.\end{Details}
\begin{Value}
A list with class \code{etest.eqdist} containing
\begin{ldescription}
\item[\code{method}] Description of test
\item[\code{statistic}] Observed value of the test statistic
\item[\code{p.value}] Approximate p-value of the test
\item[\code{sizes}] Vector of sample sizes
\item[\code{R}] Number of replicates
\item[\code{incomplete}] Argument \code{incomplete}
\item[\code{N}] Argument \code{N}
\item[\code{replicates}] Vector of replicates of the statistic
\end{ldescription}
\end{Value}
\begin{Author}\relax
Maria Rizzo \email{rizzo@math.ohiou.edu}
\end{Author}
\begin{References}\relax
Szekely, G. J. and Rizzo, M. L. (2003) Testing for Equal
Distributions in High Dimension, submitted.

Szekely, G. J. (2000) \eqn{\mathcal{E}}{E}-statistics: Energy of 
Statistical Samples, preprint.\end{References}
\begin{SeeAlso}\relax
\code{\Link{ksample.e}},
\code{\Link{print.etest.eqdist}}
\end{SeeAlso}
\begin{Examples}
\begin{ExampleCode}
 ## test if the 3 varieties of iris data (d=4) have equal distributions
 data(iris)
 eqdist.etest(iris[,1:4], c(50,50,50))
 
 ## univariate two-sample test using incomplete E-statistics
 x1 <- rnorm(200)
 x2 <- rnorm(300, .5)
 eqdist.etest(c(x1, x2), c(200, 300), incomplete=TRUE, N=100)
\end{ExampleCode}
\end{Examples}

\Header{ksample.e}{E-statistic (Energy Statistic) for Multivariate k-sample Test of Equal Distributions}
\keyword{multivariate}{ksample.e}
\keyword{htest}{ksample.e}
\keyword{nonparametric}{ksample.e}
\begin{Description}\relax
Returns the E-statistic (energy statistic)
for the multivariate k-sample test of equal distributions.
\end{Description}
\begin{Usage}
\begin{verbatim}
 ksample.e(x, sizes, distance = FALSE, ix = 1:sum(sizes), 
           incomplete = FALSE, N = 100)
\end{verbatim}
\end{Usage}
\begin{Arguments}
\begin{ldescription}
\item[\code{x}] data matrix of pooled sample
\item[\code{sizes}] vector of sample sizes
\item[\code{distance}] logical: if TRUE, x is a distance matrix
\item[\code{ix}] a permutation of the row indices of x 
\item[\code{incomplete}] logical: if TRUE, compute incomplete \eqn{\mathcal{E}}{E}-statistics
\item[\code{N}] incomplete sample size
\end{ldescription}
\end{Arguments}
\begin{Details}\relax
The k-sample multivariate \eqn{\mathcal{E}}{E}-statistic for testing equal distributions
is returned. The statistic is computed from the original pooled samples, stacked in 
matrix \code{x} where each row is a multivariate observation, or from the distance 
matrix \code{x} of the original data. The
first \code{sizes[1]} rows of \code{x} are the first sample, the next
\code{sizes[2]} rows of \code{x} are the second sample, etc.

The two-sample \eqn{\mathcal{E}}{E}-statistic proposed by Szekely and Rizzo (2003)
is the e-distance \eqn{e(S_i,S_j)}{}, defined for two samples \eqn{S_i, S_j}{}
of size \eqn{n_i, n_j}{} by
\deqn{e(S_i,S_j)=\frac{n_i n_j}{n_i+n_j}[2M_{ij}-M_{ii}-M_{jj}],
}{e(S_i, S_j) = (n_i n_j)(n_i+n_j)[2M_(ij)-M_(ii)-M_(jj)],}
where
\deqn{M_{ij}=\frac{1}{n_i n_j}\sum_{p=1}^{n_i} \sum_{q=1}^{n_j}
\|X_{ip}-X_{jq}\|,}{M_{ij} = 1/(n_i n_j) sum[1:n_i, 1:n_j] ||X_(ip) - X_(jq)||,}
\eqn{\|\cdot\|}{|| ||} denotes Euclidean norm, and \eqn{X_{ip}}{X_(ip)} denotes the p-th observation in the i-th sample.  
The k-sample  
\eqn{\mathcal{E}}{E}-statistic is defined by summing the pairwise e-distances over 
all \eqn{k(k-1)/2}{} pairs 
of samples:
\deqn{\mathcal{E}=\sum_{1 \leq i < j \leq k} e(S_i,S_j).
}{\emph{E} = sum[i<j] e(S_i,S_j).}  
Large values of \eqn{\mathcal{E}}{\emph{E}} are significant.

If \code{incomplete==TRUE}, an incomplete \eqn{\mathcal{E}}{E}-statistic (which is an
incomplete V-statistic) is computed. That is, at most
\code{N} observations from each sample are used, 
by sampling without replacement as needed.\end{Details}
\begin{Value}
The value of the multisample \eqn{\mathcal{E}}{E}-statistic corresponding to
the permutation \code{ix} is returned.\end{Value}
\begin{Note}\relax
This function computes the \eqn{\mathcal{E}}{E}-statistic only. 
For the test decision,
a nonparametric bootstrap test (approximate permutation test)
is provided by the function \code{\Link{eqdist.etest}}.
\end{Note}
\begin{Author}\relax
Maria Rizzo \email{rizzo@math.ohiou.edu}
\end{Author}
\begin{References}\relax
Szekely, G. J. and Rizzo, M. L. (2003) Testing for Equal
Distributions in High Dimension, submitted.

Szekely, G. J. (2000) \eqn{\mathcal{E}}{E}-statistics: Energy of 
Statistical Samples, preprint.\end{References}
\begin{SeeAlso}\relax
\code{\Link{eqdist.etest}}
\end{SeeAlso}
\begin{Examples}
\begin{ExampleCode}
## compute 3-sample E-statistic for 4-dimensional iris data
 data(iris)
 ksample.e(iris[,1:4], c(50,50,50))

## compute univariate two-sample incomplete E-statistic
 x1 <- rnorm(200)
 x2 <- rnorm(300, .5)
 x <- c(x1, x2)
 ksample.e(x, c(200, 300), incomplete=TRUE, N=100)
 
\end{ExampleCode}
\end{Examples}

\Header{mvnorm.e}{E-statistic (Energy Statistic) for Testing Multivariate Normality}
\keyword{multivariate}{mvnorm.e}
\keyword{htest}{mvnorm.e}
\begin{Description}\relax
Computes the E-statistic (energy statistic) for testing multivariate 
or univariate normality when parameters are estimated.
\end{Description}
\begin{Usage}
\begin{verbatim}
mvnorm.e(x)
\end{verbatim}
\end{Usage}
\begin{Arguments}
\begin{ldescription}
\item[\code{x}] matrix or vector of sample data
\end{ldescription}
\end{Arguments}
\begin{Details}\relax
If \code{x} is a matrix, each row is a multivariate observation. The
data will be standardized to zero mean and identity covariance matrix
using the sample mean vector and sample covariance matrix. If \code{x}
is a vector, the univariate statistic \code{normal.e(x)} is returned. 
If the data contains missing values or the sample covariance matrix is 
singular, NA is returned.

The \eqn{\mathcal{E}}{E}-test of multivariate normality was proposed
and implemented by Szekely and Rizzo (2004). The test statistic for 
d-variate normality is given by
\deqn{\mathcal{E} = n (\frac{2}{n} \sum_{i=1}^n E\|y_i-Z\| - 
E\|Z-Z'\| - \frac{1}{n^2} \sum_{i=1}^n \sum_{j=1}^n \|y_i-y_j\|),
}{E = n((2/n) sum[1:n] E||y_i-Z|| - E||Z-Z'|| - (1/n^2) sum[1:n,1:n]
||y_i-y_j||),}
where \eqn{y_1,\ldots,y_n}{} is the standardized sample, 
\eqn{Z, Z'}{} are iid standard d-variate normal, and
\eqn{\| \cdot \|}{|| ||} denotes Euclidean norm.\end{Details}
\begin{Value}
The value of the \eqn{\mathcal{E}}{E}-statistic for multivariate
(univariate) normality is returned.\end{Value}
\begin{Author}\relax
Maria Rizzo \email{rizzo@math.ohiou.edu}
\end{Author}
\begin{References}\relax
Szekely, G. J. and Rizzo, M. L. (2004) A New Test for 
Multivariate Normality, \emph{Journal of Multivariate Analysis},
to appear.

Rizzo, M. L. (2002). A New Rotation Invariant Goodness-of-Fit Test,
Ph.D. dissertation, Bowling Green State University.

Szekely, G. J. (1989) Potential and Kinetic Energy in Statistics, 
Lecture Notes, Budapest Institute of Technology (Technical University).\end{References}
\begin{SeeAlso}\relax
\code{\Link{normal.e}}
\end{SeeAlso}
\begin{Examples}
\begin{ExampleCode}
 
 ## compute multivariate normality test statistic for iris Setosa data
 data(iris)
 mvnorm.e(iris[1:50, 1:4])
 \end{ExampleCode}
\end{Examples}

\Header{mvnorm.etest}{E-statistic (Energy) Test of Multivariate Normality}
\keyword{multivariate}{mvnorm.etest}
\keyword{htest}{mvnorm.etest}
\begin{Description}\relax
Performs the E-statistic (energy) test of multivariate or univariate normality.
\end{Description}
\begin{Usage}
\begin{verbatim}
 mvnorm.etest(x, R = 999)
\end{verbatim}
\end{Usage}
\begin{Arguments}
\begin{ldescription}
\item[\code{x}] data matrix of multivariate sample, or univariate data vector
\item[\code{R}] number of bootstrap replicates 
\end{ldescription}
\end{Arguments}
\begin{Details}\relax
The \eqn{\mathcal{E}}{E}-test of multivariate (univariate) normality
is performed. The test is implemented by parametric bootstrap with 
\code{R} replicates. 

The definition of the \eqn{\mathcal{E}}{E}-statistic is given in the 
\code{\Link{mvnorm.e}} documentation.\end{Details}
\begin{Value}
A list with class \code{etest.mvnorm} containing
\begin{ldescription}
\item[\code{method}] Description of test
\item[\code{statistic}] Observed value of the test statistic
\item[\code{p.value}] Approximate p-value of the test
\item[\code{n}] Sample size
\item[\code{R}] Number of replicates
\item[\code{replicates}] Vector of replicates of the statistic
\end{ldescription}
\end{Value}
\begin{Author}\relax
Maria Rizzo \email{rizzo@math.ohiou.edu}
\end{Author}
\begin{References}\relax
Szekely, G. J. and Rizzo, M. L. (2004) A New Test for 
Multivariate Normality, \emph{Journal of Multivariate Analysis},
to appear.

Rizzo, M. L. (2002). A New Rotation Invariant Goodness-of-Fit Test,
Ph.D. dissertation, Bowling Green State University.

Szekely, G. J. (1989) Potential and Kinetic Energy in Statistics, 
Lecture Notes, Budapest Institute of Technology (Technical University).\end{References}
\begin{SeeAlso}\relax
\code{\Link{mvnorm.e}},
\code{\Link{print.etest.mvnorm}}
\end{SeeAlso}
\begin{Examples}
\begin{ExampleCode}
 ## test if the iris Setosa data has multivariate normal distribution
 data(iris)
 mvnorm.etest(iris[1:50,1:4])
 
 ## test a univariate sample for normality
 x <- runif(50, 0, 10)
 mvnorm.etest(x)
 \end{ExampleCode}
\end{Examples}

\Header{normal.e}{E-statistic (Energy Statistic) for Testing Univariate Normality}
\keyword{htest}{normal.e}
\begin{Description}\relax
Computes the E-statistic for testing univariate normality 
when parameters are estimated.
\end{Description}
\begin{Usage}
\begin{verbatim}
normal.e(x)
\end{verbatim}
\end{Usage}
\begin{Arguments}
\begin{ldescription}
\item[\code{x}] vector of univariate sample data
\end{ldescription}
\end{Arguments}
\begin{Details}\relax
The
data will be standardized to zero mean and unit variance
using the sample mean and sample variance. If the data contains
missing values or the sample variance is zero, NA is
returned.

The 
\eqn{\mathcal{E}}{E}-test of multivariate (and univariate)
normality was proposed and implemented by Szekely and Rizzo 
(2004). The univariate test statistic
is given by
\deqn{\mathcal{E} = n (\frac{2}{n} \sum_{i=1}^n E|y_i-Z| - E|Z-Z'| -
\frac{1}{n^2} \sum_{i=1}^n \sum_{j=1}^n |y_i-y_j|),
}{n((2/n) sum[1:n] E|y_i-Z| - E|Z-Z'| - (1/n^2) sum[1:n,1:n]
|y_i-y_j|),}
where \eqn{y_1,\ldots,y_n}{} is the standardized sample and
\eqn{Z, Z'}{} are iid standard normal variables. See 
\code{\Link{mvnorm.e}} for the multivariate statistic.\end{Details}
\begin{Value}
The value of the \eqn{\mathcal{E}}{E}-statistic for univariate
normality is returned.\end{Value}
\begin{Author}\relax
Maria Rizzo \email{rizzo@math.ohiou.edu}
\end{Author}
\begin{References}\relax
Szekely, G. J. and Rizzo, M. L. (2004) A New Test for 
Multivariate Normality, \emph{Journal of Multivariate Analysis},
to appear.

Rizzo, M. L. (2002). A New Rotation Invariant Goodness-of-Fit Test,
Ph.D. dissertation, Bowling Green State University.

Szekely, G. J. (1989) Potential and Kinetic Energy in Statistics, 
Lecture Notes, Budapest Institute of Technology (Technical University).\end{References}
\begin{SeeAlso}\relax
\code{\Link{mvnorm.e} \Link{mvnorm.e}}
\end{SeeAlso}
\begin{Examples}
\begin{ExampleCode}
 x <- rnorm(30)
 normal.e(x)
\end{ExampleCode}
\end{Examples}

\Header{poisson.m}{Mean Distance Statistic for Testing Poisson Distribution}
\keyword{htest}{poisson.m}
\begin{Description}\relax
Returns the mean distance statistic for a goodness-of-fit test of Poisson distribution with unknown parameter.
\end{Description}
\begin{Usage}
\begin{verbatim}
poisson.m(x)
\end{verbatim}
\end{Usage}
\begin{Arguments}
\begin{ldescription}
\item[\code{x}] vector of nonnegative integers, the sample data 
\end{ldescription}
\end{Arguments}
\begin{Details}\relax
The mean distance test of Poissonity was proposed and implemented by Szekely and Rizzo (2004). The test is based on the result that the sequence of expected values E|X-j|, j=0,1,2,... characterizes the distribution of the random variable X. As an application of this characterization one can get an estimator \eqn{\hat F(j)}{} of the CDF. The test statistic is a Cramer-von Mises type of distance, with M-estimates replacing the usual EDF estimates of the CDF: 
\deqn{M_n = n\sum_{j=0}^\infty (\hat F(j) - F(j\;; \hat \lambda))^2
f(j\;; \hat \lambda).}{M_n = n sum [j>=0] (\hat F(j) - F(j; \hat \lambda))^2
f(j; \hat \lambda).} See Szekely and Rizzo (2004) for the computing formula.
\end{Details}
\begin{Value}
The value of the \eqn{M}{}-statistic for testing Poisson distribution is returned.\end{Value}
\begin{Author}\relax
Maria Rizzo \email{rizzo@math.ohiou.edu}
\end{Author}
\begin{References}\relax
Szekely, G. J. and Rizzo, M. L. (2004) Mean Distance Test
of Poisson Distribution, \emph{Statistics and Probability Letters}, to appear.
\end{References}
\begin{SeeAlso}\relax
\code{\Link{poisson.mtest}}
\end{SeeAlso}
\begin{Examples}
\begin{ExampleCode}
 x <- rpois(20, 1)
poisson.m(x)
 \end{ExampleCode}
\end{Examples}

\Header{poisson.mtest}{Mean Distance Test for Poisson Distribution}
\keyword{htest}{poisson.mtest}
\begin{Description}\relax
Performs the mean distance goodness-of-fit test of Poisson distribution
with unknown parameter.
\end{Description}
\begin{Usage}
\begin{verbatim}
poisson.mtest(x, R=999)
\end{verbatim}
\end{Usage}
\begin{Arguments}
\begin{ldescription}
\item[\code{x}] vector of nonnegative integers, the sample data 
\item[\code{R}] number of bootstrap replicates 
\end{ldescription}
\end{Arguments}
\begin{Details}\relax
The mean distance test of Poissonity was proposed and implemented by Szekely and Rizzo (2004). The test is based on the result that the sequence of expected values E|X-j|, j=0,1,2,... characterizes the distribution of the random  variable X. As an application of this characterization one can get an estimator \eqn{\hat F(j)}{} of the CDF. The test statistic (see \code{\Link{poisson.m}}) is a Cramer-von Mises type of distance, with M-estimates replacing the usual EDF estimates of the CDF:
\deqn{M_n = n\sum_{j=0}^\infty (\hat F(j) - F(j\;; \hat \lambda))^2
f(j\;; \hat \lambda).}{M_n = n sum [j>=0] (\hat F(j) - F(j; \hat \lambda))^2
f(j; \hat \lambda).} The test is implemented by parametric bootstrap with \code{R} replicates.
\end{Details}
\begin{Value}
A list with class \code{etest.poisson} containing
\begin{ldescription}
\item[\code{method}] Description of test
\item[\code{statistic}] Observed value of the test statistic
\item[\code{p.value}] Approximate p-value of the test
\item[\code{n}] Sample size
\item[\code{lambda}] Sample mean
\item[\code{R}] Number of replicates
\item[\code{replicates}] Vector of replicates of the statistic
\end{ldescription}
\end{Value}
\begin{Author}\relax
Maria Rizzo \email{rizzo@math.ohiou.edu}
\end{Author}
\begin{SeeAlso}\relax
\code{\Link{poisson.m}}
\end{SeeAlso}
\begin{Examples}
\begin{ExampleCode}
 x <- rpois(20, 1)
 poisson.mtest(x)
 \end{ExampleCode}
\end{Examples}

\Header{print.etest.eqdist}{Print Multisample E-test (Energy Test) for Equal Distributions}
\keyword{print}{print.etest.eqdist}
\begin{Description}\relax
Print method for \code{etest.eqdist} object returned by the 
\code{eqdist.test} function.
\end{Description}
\begin{Usage}
\begin{verbatim}
print.etest.eqdist(x, ...)
\end{verbatim}
\end{Usage}
\begin{Arguments}
\begin{ldescription}
\item[\code{x}] an object of class \code{etest.eqdist} 
\item[\code{...}] extra arguments passed to or from other methods 
\end{ldescription}
\end{Arguments}
\begin{Author}\relax
Maria Rizzo \email{rizzo@math.ohiou.edu}
\end{Author}
\begin{SeeAlso}\relax
\code{\Link{eqdist.etest}}
\end{SeeAlso}
\begin{Examples}
\begin{ExampleCode}
## print test if the 3 varieties of iris data (d=4) have equal distributions
 data(iris)
 e <- eqdist.etest(iris[,1:4], c(50,50,50))
 print.etest.eqdist(e)
 \end{ExampleCode}
\end{Examples}

\Header{print.etest.mvnorm}{Print E-test (Energy Test) of Multivariate Normality}
\keyword{print}{print.etest.mvnorm}
\begin{Description}\relax
Print method for \code{etest.mvnorm} object returned by the 
\code{mvnorm.etest} function.
\end{Description}
\begin{Usage}
\begin{verbatim}
print.etest.mvnorm(x, ...)
\end{verbatim}
\end{Usage}
\begin{Arguments}
\begin{ldescription}
\item[\code{x}] an object of class \code{etest.mvnorm} 
\item[\code{...}] extra arguments passed to or from other methods 
\end{ldescription}
\end{Arguments}
\begin{Author}\relax
Maria Rizzo \email{rizzo@math.ohiou.edu}
\end{Author}
\begin{SeeAlso}\relax
\code{\Link{mvnorm.etest}}
\end{SeeAlso}
\begin{Examples}
\begin{ExampleCode}
 ## print E-test for 5-dimensional data
 x <- matrix(rnorm(100), nrow=20, ncol=5)
 e <- mvnorm.etest(x)
 print.etest.mvnorm(e)
 \end{ExampleCode}
\end{Examples}

\Header{print.etest.poisson}{Print Mean Distance Test for Poisson Distribution}
\keyword{print}{print.etest.poisson}
\begin{Description}\relax
Print method for \code{etest.poisson} object returned by the 
\code{poisson.mtest} function.
\end{Description}
\begin{Usage}
\begin{verbatim}
print.etest.poisson(x, ...)
\end{verbatim}
\end{Usage}
\begin{Arguments}
\begin{ldescription}
\item[\code{x}] an object of class \code{etest.poisson} 
\item[\code{...}] extra arguments passed to or from other methods 
\end{ldescription}
\end{Arguments}
\begin{Author}\relax
Maria Rizzo \email{rizzo@math.ohiou.edu}
\end{Author}
\begin{SeeAlso}\relax
\code{\Link{poisson.mtest}}
\end{SeeAlso}
\begin{Examples}
\begin{ExampleCode}
 x <- rpois(20, 1)
 e <- poisson.mtest(x)
 print.etest.poisson(e)
 \end{ExampleCode}
\end{Examples}

\end{document}
