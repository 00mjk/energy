\Header{poisson.mtest}{Mean Distance Test for Poisson Distribution}
\keyword{htest}{poisson.mtest}
\begin{Description}\relax
Performs the mean distance goodness-of-fit test of Poisson distribution
with unknown parameter.
\end{Description}
\begin{Usage}
\begin{verbatim}
poisson.mtest(x, R=999)
\end{verbatim}
\end{Usage}
\begin{Arguments}
\begin{ldescription}
\item[\code{x}] vector of nonnegative integers, the sample data 
\item[\code{R}] number of bootstrap replicates 
\end{ldescription}
\end{Arguments}
\begin{Details}\relax
The mean distance test of Poissonity was proposed and implemented by Szekely and Rizzo (2004). The test is based on the result that the sequence of expected values E|X-j|, j=0,1,2,... characterizes the distribution of the random  variable X. As an application of this characterization one can get an estimator \eqn{\hat F(j)}{} of the CDF. The test statistic (see \code{\Link{poisson.m}}) is a Cramer-von Mises type of distance, with M-estimates replacing the usual EDF estimates of the CDF:
\deqn{M_n = n\sum_{j=0}^\infty (\hat F(j) - F(j\;; \hat \lambda))^2
f(j\;; \hat \lambda).}{M_n = n sum [j>=0] (\hat F(j) - F(j; \hat \lambda))^2
f(j; \hat \lambda).} The test is implemented by parametric bootstrap with \code{R} replicates.
\end{Details}
\begin{Value}
A list with class \code{etest.poisson} containing
\begin{ldescription}
\item[\code{method}] Description of test
\item[\code{statistic}] Observed value of the test statistic
\item[\code{p.value}] Approximate p-value of the test
\item[\code{n}] Sample size
\item[\code{lambda}] Sample mean
\item[\code{R}] Number of replicates
\item[\code{replicates}] Vector of replicates of the statistic
\end{ldescription}
\end{Value}
\begin{Author}\relax
Maria Rizzo \email{rizzo@math.ohiou.edu}
\end{Author}
\begin{SeeAlso}\relax
\code{\Link{poisson.m}}
\end{SeeAlso}
\begin{Examples}
\begin{ExampleCode}
 x <- rpois(20, 1)
 poisson.mtest(x)
 \end{ExampleCode}
\end{Examples}

