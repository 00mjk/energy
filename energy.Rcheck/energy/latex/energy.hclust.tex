\Header{energy.hclust}{Hierarchical Clustering by Minimum (Energy) E-distance}
\keyword{multivariate}{energy.hclust}
\keyword{cluster}{energy.hclust}
\begin{Description}\relax
Performs hierarchical clustering on a set of Euclidean distance 
dissimilarities by minimum (energy) E-distance method.
\end{Description}
\begin{Usage}
\begin{verbatim}
    energy.hclust(dst)
\end{verbatim}
\end{Usage}
\begin{Arguments}
\begin{ldescription}
\item[\code{dst}] dissimilarity object produced by \code{dist} with
\code{method=euclidean}, or lower triangle of distance
matrix as vector in column order. If \code{dst} is a square
matrix, the lower triangle is interpreted as a vector of
distances.
\end{ldescription}
\end{Arguments}
\begin{Details}\relax
This function performs agglomerative hierarchical cluster analysis
based on the pairwise distances between sample elements in \code{dst}.
Initially, each of the n singletons is a cluster. At each of n-1 steps, the 
procedure merges the pair of clusters with minimum e-distance. 
The e-distance
between two clusters \eqn{C_i, C_j}{} of sizes \eqn{n_i, n_j}{} is given by
\deqn{e(C_i, C_j)=\frac{n_i n_j}{n_i+n_j}[2M_{ij}-M_{ii}-M_{jj}],
}{}
where
\deqn{M_{ij}=\frac{1}{n_i n_j}\sum_{p=1}^{n_i} \sum_{q=1}^{n_j}
\|X_{ip}-X_{jq}\|,}{M_{ij} = 1/(n_i n_j) sum[1:n_i, 1:n_j] ||X_(ip) - X_(jq)||,}
\eqn{\|\cdot\|}{|| ||} denotes Euclidean norm, and \eqn{X_{ip}}{X_(ip)} denotes the p-th observation in the i-th cluster.  

The return value is an object of class \code{hclust}, so \code{hclust}
methods such as print or plot methods, \code{plclust}, and \code{cutree}
are available. See the documentation for \code{hclust}.

The e-distance measures both the heterogeneity between clusters and the
homogeneity within clusters. E-clustering is particularly effective in
high dimension, and is more effective than some standard hierarchical
methods when clusters have equal means (see example below).
For other advantages see the references.\end{Details}
\begin{Value}
An object of class \code{hclust} which describes the tree produced by
the clustering process. The object is a list with components: 
\begin{ldescription}
\item[\code{merge:}] an n-1 by 2 matrix, where row i of \code{merge} describes the
merging of clusters at step i of the clustering. If an element j in the
row is negative, then observation -j was merged at this
stage. If j is positive then the merge was with the cluster
formed at the (earlier) stage j of the algorithm.
\item[\code{height:}] the clustering height: a vector of n-1 non-decreasing
real numbers (the e-distance between merging clusters)
\item[\code{order:}] a vector giving a permutation of the indices of 
original observations suitable for plotting, in the sense that a 
cluster plot using this ordering and matrix \code{merge} will not have 
crossings of the branches.
\item[\code{labels:}] labels for each of the objects being clustered.
\item[\code{call:}] the call which produced the result.
\item[\code{method:}] the cluster method that has been used (e-distance).
\item[\code{dist.method:}] the distance that has been used to create \code{dst}.
\end{ldescription}
\end{Value}
\begin{Author}\relax
Maria L. Rizzo \email{rizzo@math.ohiou.edu} and
Gabor J. Szekely \email{gabors@bgnet.bgsu.edu}
\end{Author}
\begin{References}\relax
Szekely, G. J. and Rizzo, M. L. (2003) Hierarchical Clustering
via Joint Between-Within Distances, submitted.

Szekely, G. J. and Rizzo, M. L. (2003) Testing for Equal
Distributions in High Dimension, submitted.

Szekely, G. J. (2000) \eqn{\mathcal E}{E}-statistics: Energy of
Statistical Samples, preprint.\end{References}
\begin{SeeAlso}\relax
\code{\Link{edist}} \code{\Link{ksample.e}} \code{hclust}
\end{SeeAlso}
\begin{Examples}
\begin{ExampleCode}
   ## Not run: 
   
   library(cluster)
   data(animals)
   plot(energy.hclust(dist(animals)))
   
## End(Not run)
   
   library(mva)
   data(USArrests)
   ecl <- energy.hclust(dist(USArrests))
   print(ecl)    
   plot(ecl)
   cutree(ecl, k=3)
   cutree(ecl, h=150)
   
   ## compare performance of e-clustering, Ward's method, group average method
   ## when sampled populations have equal means: n=200, d=5, two groups
   z <- rbind(matrix(rnorm(1000), nrow=200), matrix(rnorm(1000, 0, 5), nrow=200))
   g <- c(rep(1, 200), rep(2, 200))
   d <- dist(z)
   e <- energy.hclust(d)
   a <- hclust(d, method="average")
   w <- hclust(d^2, method="ward")
   list("E" = table(cutree(e, k=2) == g), "Ward" = table(cutree(w, k=2) == g),
        "Avg" = table(cutree(a, k=2) == g))
 \end{ExampleCode}
\end{Examples}

