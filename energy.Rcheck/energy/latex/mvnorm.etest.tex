\Header{mvnorm.etest}{E-statistic (Energy) Test of Multivariate Normality}
\keyword{multivariate}{mvnorm.etest}
\keyword{htest}{mvnorm.etest}
\begin{Description}\relax
Performs the E-statistic (energy) test of multivariate or univariate normality.
\end{Description}
\begin{Usage}
\begin{verbatim}
 mvnorm.etest(x, R = 999)
\end{verbatim}
\end{Usage}
\begin{Arguments}
\begin{ldescription}
\item[\code{x}] data matrix of multivariate sample, or univariate data vector
\item[\code{R}] number of bootstrap replicates 
\end{ldescription}
\end{Arguments}
\begin{Details}\relax
The \eqn{\mathcal{E}}{E}-test of multivariate (univariate) normality
is performed. The test is implemented by parametric bootstrap with 
\code{R} replicates. 

The definition of the \eqn{\mathcal{E}}{E}-statistic is given in the 
\code{\Link{mvnorm.e}} documentation.\end{Details}
\begin{Value}
A list with class \code{etest.mvnorm} containing
\begin{ldescription}
\item[\code{method}] Description of test
\item[\code{statistic}] Observed value of the test statistic
\item[\code{p.value}] Approximate p-value of the test
\item[\code{n}] Sample size
\item[\code{R}] Number of replicates
\item[\code{replicates}] Vector of replicates of the statistic
\end{ldescription}
\end{Value}
\begin{Author}\relax
Maria Rizzo \email{rizzo@math.ohiou.edu}
\end{Author}
\begin{References}\relax
Szekely, G. J. and Rizzo, M. L. (2004) A New Test for 
Multivariate Normality, \emph{Journal of Multivariate Analysis},
to appear.

Rizzo, M. L. (2002). A New Rotation Invariant Goodness-of-Fit Test,
Ph.D. dissertation, Bowling Green State University.

Szekely, G. J. (1989) Potential and Kinetic Energy in Statistics, 
Lecture Notes, Budapest Institute of Technology (Technical University).\end{References}
\begin{SeeAlso}\relax
\code{\Link{mvnorm.e}},
\code{\Link{print.etest.mvnorm}}
\end{SeeAlso}
\begin{Examples}
\begin{ExampleCode}
 ## test if the iris Setosa data has multivariate normal distribution
 data(iris)
 mvnorm.etest(iris[1:50,1:4])
 
 ## test a univariate sample for normality
 x <- runif(50, 0, 10)
 mvnorm.etest(x)
 \end{ExampleCode}
\end{Examples}

