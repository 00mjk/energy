\Header{ksample.e}{E-statistic (Energy Statistic) for Multivariate k-sample Test of Equal Distributions}
\keyword{multivariate}{ksample.e}
\keyword{htest}{ksample.e}
\keyword{nonparametric}{ksample.e}
\begin{Description}\relax
Returns the E-statistic (energy statistic)
for the multivariate k-sample test of equal distributions.
\end{Description}
\begin{Usage}
\begin{verbatim}
 ksample.e(x, sizes, distance = FALSE, ix = 1:sum(sizes), 
           incomplete = FALSE, N = 100)
\end{verbatim}
\end{Usage}
\begin{Arguments}
\begin{ldescription}
\item[\code{x}] data matrix of pooled sample
\item[\code{sizes}] vector of sample sizes
\item[\code{distance}] logical: if TRUE, x is a distance matrix
\item[\code{ix}] a permutation of the row indices of x 
\item[\code{incomplete}] logical: if TRUE, compute incomplete \eqn{\mathcal{E}}{E}-statistics
\item[\code{N}] incomplete sample size
\end{ldescription}
\end{Arguments}
\begin{Details}\relax
The k-sample multivariate \eqn{\mathcal{E}}{E}-statistic for testing equal distributions
is returned. The statistic is computed from the original pooled samples, stacked in 
matrix \code{x} where each row is a multivariate observation, or from the distance 
matrix \code{x} of the original data. The
first \code{sizes[1]} rows of \code{x} are the first sample, the next
\code{sizes[2]} rows of \code{x} are the second sample, etc.

The two-sample \eqn{\mathcal{E}}{E}-statistic proposed by Szekely and Rizzo (2003)
is the e-distance \eqn{e(S_i,S_j)}{}, defined for two samples \eqn{S_i, S_j}{}
of size \eqn{n_i, n_j}{} by
\deqn{e(S_i,S_j)=\frac{n_i n_j}{n_i+n_j}[2M_{ij}-M_{ii}-M_{jj}],
}{e(S_i, S_j) = (n_i n_j)(n_i+n_j)[2M_(ij)-M_(ii)-M_(jj)],}
where
\deqn{M_{ij}=\frac{1}{n_i n_j}\sum_{p=1}^{n_i} \sum_{q=1}^{n_j}
\|X_{ip}-X_{jq}\|,}{M_{ij} = 1/(n_i n_j) sum[1:n_i, 1:n_j] ||X_(ip) - X_(jq)||,}
\eqn{\|\cdot\|}{|| ||} denotes Euclidean norm, and \eqn{X_{ip}}{X_(ip)} denotes the p-th observation in the i-th sample.  
The k-sample  
\eqn{\mathcal{E}}{E}-statistic is defined by summing the pairwise e-distances over 
all \eqn{k(k-1)/2}{} pairs 
of samples:
\deqn{\mathcal{E}=\sum_{1 \leq i < j \leq k} e(S_i,S_j).
}{\emph{E} = sum[i<j] e(S_i,S_j).}  
Large values of \eqn{\mathcal{E}}{\emph{E}} are significant.

If \code{incomplete==TRUE}, an incomplete \eqn{\mathcal{E}}{E}-statistic (which is an
incomplete V-statistic) is computed. That is, at most
\code{N} observations from each sample are used, 
by sampling without replacement as needed.\end{Details}
\begin{Value}
The value of the multisample \eqn{\mathcal{E}}{E}-statistic corresponding to
the permutation \code{ix} is returned.\end{Value}
\begin{Note}\relax
This function computes the \eqn{\mathcal{E}}{E}-statistic only. 
For the test decision,
a nonparametric bootstrap test (approximate permutation test)
is provided by the function \code{\Link{eqdist.etest}}.
\end{Note}
\begin{Author}\relax
Maria Rizzo \email{rizzo@math.ohiou.edu}
\end{Author}
\begin{References}\relax
Szekely, G. J. and Rizzo, M. L. (2003) Testing for Equal
Distributions in High Dimension, submitted.

Szekely, G. J. (2000) \eqn{\mathcal{E}}{E}-statistics: Energy of 
Statistical Samples, preprint.\end{References}
\begin{SeeAlso}\relax
\code{\Link{eqdist.etest}}
\end{SeeAlso}
\begin{Examples}
\begin{ExampleCode}
## compute 3-sample E-statistic for 4-dimensional iris data
 data(iris)
 ksample.e(iris[,1:4], c(50,50,50))

## compute univariate two-sample incomplete E-statistic
 x1 <- rnorm(200)
 x2 <- rnorm(300, .5)
 x <- c(x1, x2)
 ksample.e(x, c(200, 300), incomplete=TRUE, N=100)
 
\end{ExampleCode}
\end{Examples}

