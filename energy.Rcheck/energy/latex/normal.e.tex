\Header{normal.e}{E-statistic (Energy Statistic) for Testing Univariate Normality}
\keyword{htest}{normal.e}
\begin{Description}\relax
Computes the E-statistic for testing univariate normality 
when parameters are estimated.
\end{Description}
\begin{Usage}
\begin{verbatim}
normal.e(x)
\end{verbatim}
\end{Usage}
\begin{Arguments}
\begin{ldescription}
\item[\code{x}] vector of univariate sample data
\end{ldescription}
\end{Arguments}
\begin{Details}\relax
The
data will be standardized to zero mean and unit variance
using the sample mean and sample variance. If the data contains
missing values or the sample variance is zero, NA is
returned.

The 
\eqn{\mathcal{E}}{E}-test of multivariate (and univariate)
normality was proposed and implemented by Szekely and Rizzo 
(2004). The univariate test statistic
is given by
\deqn{\mathcal{E} = n (\frac{2}{n} \sum_{i=1}^n E|y_i-Z| - E|Z-Z'| -
\frac{1}{n^2} \sum_{i=1}^n \sum_{j=1}^n |y_i-y_j|),
}{n((2/n) sum[1:n] E|y_i-Z| - E|Z-Z'| - (1/n^2) sum[1:n,1:n]
|y_i-y_j|),}
where \eqn{y_1,\ldots,y_n}{} is the standardized sample and
\eqn{Z, Z'}{} are iid standard normal variables. See 
\code{\Link{mvnorm.e}} for the multivariate statistic.\end{Details}
\begin{Value}
The value of the \eqn{\mathcal{E}}{E}-statistic for univariate
normality is returned.\end{Value}
\begin{Author}\relax
Maria L. Rizzo \email{rizzo@math.ohiou.edu} and
Gabor J. Szekely \email{gabors@bgnet.bgsu.edu}
\end{Author}
\begin{References}\relax
Szekely, G. J. and Rizzo, M. L. (2004) A New Test for 
Multivariate Normality, \emph{Journal of Multivariate Analysis},
\url{http://dx.doi.org/10.1016/j.jmva.2003.12.002}.

Rizzo, M. L. (2002). A New Rotation Invariant Goodness-of-Fit Test,
Ph.D. dissertation, Bowling Green State University.

Szekely, G. J. (1989) Potential and Kinetic Energy in Statistics, 
Lecture Notes, Budapest Institute of Technology (Technical University).\end{References}
\begin{SeeAlso}\relax
\code{\Link{mvnorm.e}}, \code{\Link{mvnorm.etest}}
\end{SeeAlso}
\begin{Examples}
\begin{ExampleCode}
 x <- rnorm(30)
 normal.e(x)
\end{ExampleCode}
\end{Examples}

